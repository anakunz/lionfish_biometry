%% Submissions for peer-review must enable line-numbering 
%% using the lineno option in the \documentclass command.
%%
%% Preprints and camera-ready submissions do not need 
%% line numbers, and should have this option removed.
%%
%% Please note that the line numbering option requires
%% version 1.1 or newer of the wlpeerj.cls file, and
%% the corresponding author info requires v1.2

\documentclass[fleqn,10pt,lineno]{wlpeerj} % for journal submissions
% \documentclass[fleqn,10pt]{wlpeerj} % for preprint submissions

\title{Invasive lionfish present region-wide variation of allometric growth in
the Western Atlantic}

\author[1]{Juan Carlos Villaseñor-Derbez\(^1\), Sean Fitzgerald}
\affil[1]{Bren School of Environmental Sciences and Management, University of
California Santa Barbara, Santa Barbara, California, USA}
\corrauthor[1]{Juan Carlos Villaseñor-Derbez}{jvillasenor@bren.ucsb.edu}

% \keywords{A, B, C}

\begin{abstract}
Lionfish (\emph{Pterois volitans/miles}) are an invasive species in the
Western Atlantic and the Caribbean. Improving management of invasive
lionfish populations requires accurate estimates of total biomass, which
in turn depend on accurate definition of the length-weight relationship.
Here, we reviewed published length-weight relationships of lionfish
taken from throughout their invasive range and found that lionfish of
equal lengths have lower body mass in the Caribbean than in the Atlantic
or Gulf of Mexico. Additionally, we report a new pair of length-weight
parameters (\(a = 3.2056 \times 10^{-6}; b = 3.235\)) for organisms
sampled in the central Mexican Caribbean region. The substantial spatial
variation in length-weight parameters highlights the importance of using
site-specific information when estimating lionfish biomass using length
observations. These findings can have major implications for management
in terms of predicting effects on local ecosystems, evaluating the
effectiveness of removal programs, or estimating biomass available for
harvest.
\end{abstract}

\begin{document}

\flushbottom
\maketitle
\thispagestyle{empty}

\section*{Introduction}

Lionfish (\emph{Pterois volitans/miles} complex) are an invasive species
in the western Atlantic and the Caribbean, likely introduced through
liberation of aquarium-kept organisms \citep{betancurr_2011}. They are
the first invasive marine vertebrates established along the North
Atlantic Caribbean coasts
\citep{schofield_2009,schofield_2010,sabidoitza_2016} and their presence
has been labeled as a major marine invasion because they threaten local
biodiversity, spread rapidly, and are difficult to manage
\citep{hixon_2016}. Invasive lionfish are primarily studied in coral
reef ecosystems, where their impacts are far-reaching. For example,
field experiments by \citet{albins_2008} showed that lionfish
establishment led to reduced recruitment of native fishes by nearly 80\%
over a five week period in Florida, and increased lionfish biomass
coincided with a 65\% reduction in the biomass of native prey fishes
along Bahamian coral reefs in just two years \citep{green_2012}.
Lionfish have also established invasive populations in other habitats
such as estuaries, mangroves , hard-bottomed areas , and mesophotic
reefs
\citep{jud_2011,barbour_2010,muoz_2011,andradibrown_2017,claydon_2012}.

A substantial amount of research describes lionfish feeding ecology from
North Carolina to Costa Rica
\citep{muoz_2011,morris_2009,cote_2013,dahl_2014,valdezmoreno_2012,villaseorderbez_2014,hackerott_2017,sandel_2015}.
A meta-analysis by \citet{peake_2018} showed that invasive lionfish prey
on at least 167 different species across the tropical and temperate
North Atlantic. Their feeding behavior and high consumption rates can
reduce recruitment and population sizes of native reef-fish species, and
can further endanger reef fish
\citep{albins_2008, green_2012,rocha_2015}. (However, see
\citet{hackerott_2017} for a case where there was no evidence that
lionfish affected the density, richness, or community composition of
prey fishes). Major efforts have been made to understand the possible
impacts of the invasion by tracking the spread of established lionfish
populations through time \citep{schofield_2009,schofield_2010} and by
predicting invasion ranges under future climates \citep{grieve_2016}.
Trophic impacts of lionfish can be minimized if local lionfish biomass
is controled by by culling \citep{ariasgonzalez_2011}. However, the
feasability of controling local populations remains a topic of debate
\citep{cote_2014}.

Governments and non-profit organizations have sought to reduce lionfish
densities through removal programs and incentivizing its consumption
\citep{chin_2016}. In some cases, these have shown to significantly
reduce --but not quite eliminate-- lionfish abundances at local scales
\citep{sandel_2015,chin_2016,deleon_2013}. In addition, culling programs
can help stabilize or grow native prey fish populations
\citep{cote_2014}. Complete eradication of lionfish through fishing is
unlikely because of their rapid recovery rates and ongoing recruitment
to shallow-water areas from their persistent populations in mesophotic
coral ecosystems \citep{barbour_2011,andradibrown_2017}. However,
promoting lionfish consumption might create a level of demand capable of
sustaining a stable fishery, which can help control shallow-water
populations while providing alternative livelihoods and avoiding further
impacts to local reef biota \citep{chin_2016}.

The feasibility of establishing fisheries through lionfish removal
programs has been extensively evaluated through field observations and
empirical modeling
\citep{barbour_2011,morris_2011,deleon_2013,johnston_2015,sandel_2015,chin_2016,usseglio_2017}.
One contributing factor to the success of many removal programs is the
sedentary nature of adult lionfish \citep{jud_2012}. Culling programs
are effective in reducing adult populations largely because lionfish
exhibit high levels of site fidelity and rarely leave their home range
in most cases \citep{Fishelson_1997,cote_2014,kochzius_2005} -- But see
\citet{andradibrown_2017} for cases when deep-water populations maintain
shallow-water populations. Fish with this sedentary behavior are likely
to exhibit high levels of spatial variation in important life history
characterstics such as growth or natural mortality rate
\citep{hutchinson_2008,wilson_2012}. The importance of considering
spatial heterogeneity is well-documented in terms of assessing and
managing sedentary species \citep{gunderson_2008,guan_2013}, and such
variation should be accounted for when evaluating the feasibility of
establishing lionfish fisheries as well.

Empirical modeling efforts examining the feasibility of establishing
fisheries for lionfish involve modeling changes in biomass in response
to changes in mortality (\emph{i.e.} culling). A common way to model
this is via length-structured population models, where fish lengths are
converted to weight in order to calculate total biomass
\citep{cote_2014,barbour_2011,andradibrown_2017}. The length-weight
relationship is therefore an essential component of these models, but
this relationship can vary across regions as a response to biotic
(e.g.~local food availability) and abiotic (e.g.~water temperature)
conditions \citep{johnson_2016}. Literature suggests that site-specific
parameters are necessary in order to accurately estimate biomass when
length-weight relationships are spatially variable , and this
variability becomes increasingly important when estimating the potential
effectiveness of (and resources needed for) lionfish culling programs or
when identifying total biomass available for harvest by fishers
\citep{barbour_2011,morris_2011,johnston_2015,chin_2016,cote_2014}. In
addition to environmentally-driven spatial variation, genetic analysis
of invasive lionfish suggest biological differences due to the existence
of two genetically distinct subpopulations between the northwest
Atlantic and the Caribbean \citep{betancurr_2011}. To date, no studies
have examined region-wide differences in length-weight parameters
despite the large number of studies reporting this relationship for
lionfish.

The objective of this paper is to describe the spatial pattern of
length-weight relationships of lionfish in the Caribbean and Western
Atlantic and evaluate the implications of these spatial differences.
Length-weight relationships for lionfish exist for North Carolina,
Northern and Southern Gulf of Mexico, the Southern Mexican Caribbean,
Bahamas, Little Cayman, Jamaica, Bonaire, Puerto Rico, and Costa Rica
\citep{barbour_2011,fogg_2013,dahl_2014,aguilarperera_2016,sabidoitza_2016,sabidoitz_2016,darling_2011,edwards_2014,chin_2016,deleon_2013,toledohernndez_2014,sandel_2015}.
This study also provides the first length-weight relationship for the
central Mexican Caribbean.

\clearpage

\section*{Materials and Methods}

We reviewed 12 published studies and obtained 17 length-weight
relationships for the North Atlantic (n = 1), Gulf of Mexico (n = 7,),
and Caribbean (n = 10, Table \ref{tab:all_params}, Fig
\ref{fig:all_allo}). We collected information on sampling methods, sex
differentiation, location, and depth ranges from each study when
available, and assumed both genders were included in a study if gender
was unspecified. Two studies reported parameters for genders combined an
separated \citep{aguilarperera_2016,fogg_2013}, while the rest presented
pooled results. Reviewed studies presented information for organisms
obtained at depths between 0.5 and 57 m. Three studies explicitly stated
that their organisms were sampled with pole spears
\citep{aguilarperera_2016,chin_2016,dahl_2014,sabidoitz_2016}, and five
studies mentioned that some of their organisms were obtained with pole
spears (or other type of harpoon) but also hand-held nets or fish traps
\citep{sandel_2015,barbour_2011,fogg_2013,edwards_2014,sabidoitza_2016,sabidoitz_2016,toledohernndez_2014},
and two studies did not specify how organisms were sampled
\citep{deleon_2013,darling_2011}. \citet{fogg_2013} use spine-less
weight in the length-weight relationship estimation, and thus their
parameters likely underestimate whole wieght. Since no spine-less to
whole weight conversions were available, these parameters were taken as
reported.

We also collected data from 10 sampling sites along the central Mexican
Caribbean coast in 2010 (Table S1). Sampling locations included wall and
carpet reefs at depths between 5.7 m and 38.1 m. All observed lionfish
(n = 109) were collected using hand nets and numbered collection
bottles. The use of hand nets prevented any weight loss due to bleeding
and allowed better representation of small sizes by eliminating gear
selectivity. Organisms were euthanized via pithing and Total Length (TL;
mm) and Total Weight (TW; g) were recorded.

\begin{figure}
\centering
\includegraphics{Manuscript_files/figure-latex/unnamed-chunk-1-1.pdf}
\caption{\label{fig:map}Locations where allometric growth parameters of
lionfish (\emph{Pterois spp}) have been reported. Circle sizes indicate
sample size from each study, colors indicate the \(b\) coefficient from
Eq. \ref{eq:allometric}.}
\end{figure}

The weight at length relationship for lionfish in the central Mexican
Caribbean was calculated with the allometric growth function:

\begin{equation}
\label{eq:allometric}
TW = aTL^b
\end{equation}

Where \(a\) is the ponderal index and \(b\) is the scaling exponent or
allometric parameter. When \(b = 3\), it is said that the organism
exhibits a perfect isometric growth. Transforming this equation via
base-10 logarithms:

\begin{equation}
\label{eq:log-alo}
log_{10}(TW) = b\times log_{10}(TL) + log_{10}(a)
\end{equation}

This can be simplified and re-written as:

\begin{equation}
\label{eq:log-alo-trans}
Y = bX + c
\end{equation}

Where \(Y = log_{10}(TW)\), \(X = log_{10}(TL)\), and
\(c = log_{10}(a)\). The coefficients (\(c\) and \(b\)) were estimated
with an Ordinary Least Squares Regression and heteroskedastic-robust
standard error correction \citep{zeileis_2004}. The \(b\) coefficient
was tested against the null hypothesis of isometric growth (\emph{i.e.}
\(H_0: b = 3\)). Coefficients were tested with a two-tailed Student's t,
and the significance of the regression was corroborated with an F-test.

Some of the reviewed studies inconsistently defined \(a\) as either the
ponderal index from Eq. \ref{eq:allometric} or the y-intercept (\(c\))
from Eq. \ref{eq:log-alo-trans}. Other studies incorrectly reported
parameters as mm-to-g conversions when they were in fact cm-to-g
conversions. We standardized each study by converting coefficients and
report all parameters as TL(mm) to TW (gr) conversions. Locations where
allometric studies have been performed are shown in Figure \ref{fig:map}
and Table \ref{tab:all_params}.

Combining the length-weight parameters extracted from the literature and
the additional pair calculated here, we obtain a total of 18 pairs. We
use the central Mexican Caribbean as a case study of how the use of
\emph{ex situ} parameters influences the accuracy of weight estimates
for lionfish. We estimated TW from the TL observations we collected in
the central Mexican Caribbean (n = 109) using each of the 18 parameter
pairs and divided predicted weights by known observed weights to obtain
a simple measure of over- or underestimation. Difference in mean weight
ratios across the different parameter pairs were tested with a one-way
analysis of variance (ANOVA). All analyses were performed in R version
3.5.0 \citep{rcore_2018} . Raw data and code used in this work are
available at dryad.org.

\section*{Results}

Inserting TL observations for the central Mexican Caribbean into
\ref{eq:log-alo} Eq. 2 and converting to \ref{eq:log-alo-trans} resulted
in the coefficient values \(b = 3.2347391\) and \(c = -5.4940866\)
(\(R^2 = 0.977\), F(df = 1; 107) = 6928.67, \(p < 0.001\)). The
length-weight coefficients estimated in this study were within the range
identified by studies in other regions (Table \ref{tab:all_params}). The
significant difference between this allometric factor (\(b\)) and the
isometric growth factor \(b = 3\) (\(t(107) = 6.04; p<0.001\)) indicated
that lionfish present allometric growth, which is consistent with other
studies. Figure \ref{fig:l-w-carib} shows the relationship between TL
and TW for this region, and more information on model fit is presented
in Table S2.

Parameters from models fit to males or females exclusively tend to have
a higher steepness (\emph{i.e.} higher allometric parameter), with mean
\(\pm\) standard deviation values of \(b = 3.27 \pm 0.06\) and
\(b = 3.31 \pm 0.23\) for males and females respectively, compared to
parameters from models for pooled genders with a mean \(\pm\) standard
deviation value of \(b = 3.14 \pm 0.20\). In the case of the ponderal
index (\(a\)) and its \(log_{10}\) transformation (\(c\)), values were
higher for parameters for pooled genders. Figure \ref{fig:all_allo}
shows the length-weight relationships with parameters from all studies.

There were significant differences in our predicted weights for the
central Mexican Caribbean when using the different pairs of parameters
(\(F(df = 15; 1728) = 38.26; p < 0.001\)). The lowest weight estimates
resulted from using the allometric parameters from Banco Chinchorro in
the Caribbean \citet{sabidoitz_2016}, and the highest weight estimates
came from the Northern Atlantic\citet{barbour_2011}. The calculated
ratio of predicted-to-observed weight ranged from 0.80 \(\pm\) 0.19 to
1.76 \(\pm\) 0.50 (mean \(\pm\) SD). Predicted-to-observed weight ratios
are presented in Figure \ref{fig:bio_ratio}. Spine-less weight
parameters from \citet{fogg_2013} still produced predicted-to-observed
weight rations \textgreater{} 1.

\begin{figure}
\centering
\includegraphics{Manuscript_files/figure-latex/unnamed-chunk-4-1.pdf}
\caption{\label{fig:l-w-carib}Length-weight relationship for 109
lionfish sampled in the central Mexican Caribbean. Points indicate
samples, dashed black line indicates curve of best fit, marginal plots
represent the density distribution of each variable.}
\end{figure}

\begin{table}

\caption{\label{tab:unnamed-chunk-5}\label{tab:all_params}Summary of 18 allometric growth parameters available for lionfish in the invaded range from peer-reviewed literature and this study. All parameters have been adjusted to convert from millimeters to grams. n = Sample size, Sex specifies whether data was presented for Females (F), Males (M), or both genders combined (B), a = scaling parameter for Eq. 1 (presented in $\times 10^{-5}$), c = y-intercept for Eq. 3, b = exponent or slope for Eq. 1 or Eq. 3, respectively. The $R^2$ column indicates reported model fit.}
\centering
\begin{tabular}[t]{llllrrll}
\toprule
Region & Sex & n & a & b & c & \$R\textasciicircum{}2\$ & Reference\\
\midrule
Caribbean & B & 458 & 3.6 & 2.81 & -4.44 & - & Sandel et al., 2015\\
Caribbean & B & 419 & 2.8 & 2.85 & -4.56 & 0.8715 & Chin et al., 2016\\
Caribbean & B & 1450 & 2.3 & 2.89 & -4.64 & 0.96 & de Leon et al., 2013\\
Caribbean & B & 1887 & 0.3 & 3.24 & -5.52 & 0.97 & Edwards et al., 2014\\
Caribbean & B & - & 0.25 & 3.29 & -5.60 & - & Darling et al., 2011\\
\addlinespace
Caribbean & B & 2143 & 0.52 & 3.18 & -5.28 & 0.9907 & Sabido-Itza et al., 2016\\
Caribbean & B & 227 & 0.8 & 3.11 & -5.10 & 0.958 & Toledo-Hernández et al., 2014\\
Caribbean & B & 449 & 0.23 & 3.25 & -5.64 & 0.97 & Sabido-Itza et al., 2016b\\
Caribbean & B & 368 & 0.32 & 3.19 & -5.50 & 0.98 & Sabido-Itza et al., 2016b\\
Caribbean & B & 109 & 0.32 & 3.23 & -5.49 & 0.9766 & This study\\
\addlinespace
GoM & B & 934 & 0.21 & 3.34 & -5.68 & 0.98 & Dahl \& Patterson, 2014\\
GoM & B & 472 & 0.29 & 3.30 & -5.54 & 0.95 & Aguilar-Perera \& Quijano-Puerto, 2016\\
GoM & F & 67 & 0.12 & 3.47 & -5.93 & 0.95 & Aguilar-Perera \& Quijano-Puerto, 2016\\
GoM & M & 59 & 0.42 & 3.23 & -5.38 & 0.95 & Aguilar-Perera \& Quijano-Puerto, 2016\\
GoM & B & 582 & 0.14 & 3.43 & -5.86 & 0.99 & Fogg et al., 2013\\
\addlinespace
GoM & M & 119 & 0.27 & 3.31 & -5.57 & 0.97 & Fogg et al., 2013\\
GoM & F & 115 & 0.68 & 3.14 & -5.17 & 0.94 & Fogg et al., 2013\\
NorthAtlantic & B & 774 & 2.9 & 2.89 & -4.54 & - & Barbour et al.,2011\\
\bottomrule
\end{tabular}
\end{table}

\begin{figure}
\centering
\includegraphics{Manuscript_files/figure-latex/unnamed-chunk-6-1.pdf}
\caption{\label{fig:all_allo}Length-weight relationships (n = 18) for 12
studies and this study. Colors indicate studies from which the
parameters were extracted. Solid lines indicate that the fit was
performed for males and females pooled together. Dotted lines indicate
that the regression was performed on females, and dashed lines indicate
it was performed for males. The dashed black line represents the
relationship estimated in this study.}
\end{figure}

\begin{figure}
\centering
\includegraphics{Manuscript_files/figure-latex/unnamed-chunk-7-1.pdf}
\caption{\label{fig:bio_ratio}Violin plot of predicted to observed
weight ratios for 18 pairs of allometric parameters. Red and blue
circles indicate median and mean values, respectively. Like letters
indicate values that do not differ significantly (Tukey's HSD; p
\textless{} 0.05).}
\end{figure}

\clearpage

\section*{Discussion}

We detected substantial differences in weight-at-length between
organisms from the Caribbean and Gulf of Mexico / North-Western
Atlantic. Weight estimates using parameters from the Gulf of Mexico and
North-Western Atlantic were higher on average than those from the
Caribbean. The average predicted-to-observed weight ratios from these
three regions were insert 1.24 \(\pm\) 0.309, 1.76 \(\pm\) 0.496, and
Caribbean 1.17 \(\pm\) 0.398, respectively. These length-weight
differences mirror similar findings of regional variability in
age-at-length relationships of lionfish across both their invaded and
native regions \citep{pusack_2016}. These differences may be driven by
genetic variation or by organisms being exposed to distinct
environmental conditions. For example, \citet{betancurr_2011} used
mitochondrial DNA to demonstrate the existence of two distinct
population groups, identified as the ``Caribbean group'' and ``Northern
Group'', and \citet{fogg_2015} alternatively suggested that
age-at-length differences may be driven by climate. Differences in
weight-at-length could also reflect differences in energy input
(\emph{i.e.} in some regions, lionfish eat more) or differential usage
of this energy (\emph{e.g.} regional differences in predator abundances
lead to different usage of energy), or a combination of both. Future
research is needed to determine which processes are at work here.

Differences in length-weight relationships have traditionally been
highlighted as potential pitfalls to fishery management. For example,
\citet{wilson_2012} show that small-scale variations in length-at-age
and fishing mortality in other Scorpaeniformes (\emph{Sebastes
rastrelliger}) translate to differential landings, effort, and catch per
unit effort in the live fish fishery of California, and that these
differences must be taken into account in management plans. The lionfish
case poses the opposite scenario, where the manager desires to
erradicate species. To accurately gauge both the effectiveness of
lionfish removal efforts and the resources needed to successfully manage
an invasion, we must acknowledge and understand regional biological
differences in important variables such as allometric growth parameters.

The results presented here have major implications for management. For
example, \citet{edwards_2014} simulated a lionfish culling program under
two scenarios, one using length-at-age and length-to-weight parameters
from North Carolina and one using parameters from Little Cayman. They
showed that using the different parameters caused up to a four-year
difference in the time required for the simulated lionfish population to
recover to 90\% of its initial biomass after removals ceased. Here, we
show that using one set of length-weight parameters versus another for a
given length can result in up to a threefold TW overestimation. These
differences become especially important when allocating resources for
removal programs, incentivizing its fishery as alternative livelihoods,
or estimating ecosystem impacts. Research efforts focused on invasive
lionfish populations need to use parameters calculated for their region
to the extent possible, or at least use reasonable sets of different
parameters that provide upper and lower bounds in their results. This
work additionally highlights the need for more basic research that
furthers our understanding of the invasive lionfish.

\section*{Aknowledgements}

The authors would like to thank thank Nils Van Der Haar and Michael
Doodey from Dive Aventuras as well as Guillermo Lotz-Cador who provided
help to collect samples.

\bibliography{references}

\end{document}